\newpage

\centerline{\fangsong\bf\zihao{-2}{Research on Content-Aware Collaborative Filtering }}

\addcontentsline{toc}{section}{Abstract(Key words)}

\vskip 20bp

\hspace{4bp} {\zihao{-4}\textbf{【 Abstract】}} 
Pairwise learning algorithms are a vital technique
for personalized ranking with implicit feedback. They usually
assume that each user is more interested in items which have
been selected by the user than remaining ones. This pairwise
assumption usually derives massive training pairs. To deal with
such large-scale training data, the learning algorithms are usually
based on stochastic gradient descent with uniformly drawn pairs.
However, the uniformly sampling strategy often results in slow
convergence. In this paper, we first uncover the reasons of
slow convergence. Then, we associate contents of entities with
characteristics of dataset to develop an adaptive item sampler
for drawing informative training data. In this end, to devise a
robust personalized ranking method, we accordingly embed our
sampler into Bayesian Personalized Ranking (BPR) framework,
and further propose a Content-aware and Adaptive Bayesian
Personalized Ranking (CA-BPR) method, which can model both
contents and implicit feedbacks in a unified learning process. The
experimental results show that our adaptive item sampler can indeed improve recommendation performance.

\vskip 10bp

\hspace{5bp}{\zihao{-4}\textbf{【 Keywords】}}
Recommendation System; Collaborative Filtering; Adaptive Sampling


\vskip 20bp

\begin{flushright}
	\kaishu指导教师:\ 潘微科 \hspace{3cm}{ }
\end{flushright}

\label{lastpage}%%%%显示总页数