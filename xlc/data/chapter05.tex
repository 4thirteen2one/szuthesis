\section{融合内容信息的适应性BPR}
在上述章节中, 我们阐述了如何通过一个适应性采样策略加快BPR的学习, 同时通过仅考虑隐式反馈学习了the latent factors of entities。 不过, 在现实世界的推荐系统中, 很可能没有足够的协同信息, 比如, 新的物品可能会在任何时间被加入到推荐系统中。因此, 我们提出一个更为全面的个性化推荐方法:Content-aware and Adaptive Bayesian Personalized Ranking, 它基于上面所提出的适应性采样策略, 同时将隐式反馈与内容信息融合入一个统一的推荐框架中.


\subsection{Learning Content-aware Mappings}
我们首先正提出一个对于学习content-aware mappings的一个非监督解决方案。用矩阵$A^e = \left[a_1^e,a_2^e,a_3^e,\dots\right]$来表示content features of entities.然后我们提出对于学习content-aware mappings的目标函数:
\begin{equation}
L_{content} = \| A^eW^e - Y^e\|_F^2
\end{equation}
这里的$W^e \in \mathbb{R}^{d^e \times k}$表示映射矩阵(mapping matrix), $k$表示latent vectors的维度。




\subsection{Parameter Inference of CA-BPR}
通常来讲, 由于缺乏监督信息(supervised information),在公式所表述的优化问题并无确定解法。不过, 根据子空间的研究, 我们可以从隐式反馈中学习一个latent matrix $\widetilde{Y^e}$, 并用$\widetilde{Y^e}$近似代替$Y^e$. 因此, 将$\widetilde{Y^e}$代替$Y^e$代入公式, 那么目标函数变为: 
\begin{equation}
L_{content} = \| A^eW^e - \widetilde{Y^e}\|_F^2
\end{equation}

使用$\widetilde{Y^e}$近似代替$Y^e$不仅能够优化目标函数, 同时还能够一起学习包含协同信息与内容信息的 $W^{e}$. 因此, 算法总体的目标函数如下:
\begin{equation}
%分隔一个过长的公式分行显示使用split环境
\begin{split}
arg \min_{\substack{\Theta, W}} L_{feedback}+L_{content}  = 
& - \sum_{\left(m,i,j\right) \in D_s} \ln f \left( r_{mij}\right) + \lambda\|\theta\|^2\\
& + \|A^eW^e-Y^e\|^2_F + \frac 12 \sum_{e\in \{u,v\}}\lambda^e\|W^e\|^2_F 
\end{split}
\end{equation}
这里的$r_{mij} = r_{mi} -r_{mj}$。
为了学习在公式中的参数$Y^u$, $Y^v$, $W^u$, $W^v$, 在每轮迭代中, 当我们更新latent factor matrix $Y^e$, 将矩阵$W^e$认为是一个常量(constant), 并将$L_{content}$视作一个正则化项(regularizer)。那么, 对于一个任意 latent parameter $\theta$ 的梯度如下:
\begin{equation}
\begin{split}
\frac{\partial L}{\partial\theta} = 
& \sum_{\left(m,i,j\right) \in D_s}\left(f\left(r_{mij}\right) - 1\right) 
\frac{\partial \left(r_{mij}\right)}{\partial \theta}\\
& + \frac{\partial \sum_{e \in \{u,v\} } \lambda^e \left(\|A^eW^e-Y^e\|_F^2\right)}{\partial \theta}   + \lambda \theta 
\end{split}
\end{equation}
对于参数$\theta$的更新公式为: $\theta = \theta - \gamma \frac{\partial L}{\partial \theta}$, 这里的$\gamma$为学习率(learning rate)。另一方面, 对于一个latent factor matrix $Y^e$, 将$Y^e$视为伪标签(pseudo labels), 并视$L_{feedback}$为常量。因此对目标求偏导:
\begin{equation}
\frac{\partial L}{\partial W^e} = \left(A^e\right)^T\left(A^eW^e-Y^e\right) + \lambda^e W^e
\end{equation} 
令$\frac{\partial L}{\partial W^e} = 0$, 那么对于$W^e$的更新公式则演变为:
\begin{equation}
\label{equ:W}
W^e = \left(\left(A^e\right)^TA^e + \lambda^e\mathbb{E}\right)A^eY^e
\end{equation}
这里的$\mathbb{E} \in \mathbb{R}^{k\times k}$表示一个单位矩阵。

总言之, 对于CA-BPR的参数学习如算法2所示.
\IncMargin{1em}
\begin{algorithm}
	\SetAlgoNoLine %不要算法中的竖线
	
	\SetKwInOut{Input}{\textbf{输入}}\SetKwInOut{Output}{\textbf{输出}}
	
	\Input{
		\\
		The observed user-item pair set $S$\;\\
		The feature matrix of items $F$\;\\
		The content features entities $A := \{A^u,A^v\}$\;\\}
	\Output{
		\\
		$\Theta \  := \{Y^u,Y^v\}$\;\\
		$W := \{W^u,W^v\}$\;\\}
	\BlankLine
	initialize the model parameter $\Theta$ and $W$ with uniform $\left(-\sqrt{6}/{k},\sqrt{6}/{k}\right)$\;
	standarized $\Theta$\;
	Initialize the popularity of categories $\rho$ randomly\;
	\Repeat
		{\text{convergence}}
		{Draw a triple $\left(m,i,j\right)$ with 算法\ref{al2}\;
			\For {each latent vector $\theta \in \Theta$}{
				$\theta \leftarrow \theta - \eta\frac{\partial L}{\partial \theta}$
			}
			
			\For {each $W^e \in W$}{
				Update $W^e$ with the rule defined in Eq.\ref{equ:W}\;
			}	
		}
		
		
	\caption{Learning paramters for BPR\label{al3}}
	
\end{algorithm}
\DecMargin{1em}

\subsection{本章小结}
本章通过学习了一个mapping矩阵利用了内容信息,同时将本文所研究的适应性采样策略融合BPR的推荐框架中,提出了CA-BPR推荐算法。