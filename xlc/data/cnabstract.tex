\newpage

\centerline{\fangsong\bf\zihao{-2}{融合内容信息的单类协同过滤推荐算法研究}}
\addcontentsline{toc}{section}{摘要(关键词)}%加入目录


\vskip 1cm

\begin{center}
	\kaishu
	\hspace{2cm}计算机与软件学院计算机科学与技术专业 \quad 徐留成 
	\vspace{5bp}
	\newline
	学号:2012080173
\end{center}

\vskip 10bp

{
\kaishu	
\hspace{5bp}{\zihao{-4}\textbf{【摘要】}} 
对于基于隐式反馈的个性化推荐排名而言,pairwise learning是一个非常重要的技术手段。pair learning algorithms 通常假设一个用户相比于未选择过的物品会更感兴趣于已选择过的物品。这种假设在推荐算法中会衍生出大量的training pairs。而为了应对大规模的数据集,这些推荐算法往往都是基于均匀采样的随机梯度下降方法进行求解。不过,这种采取均匀采样的策略经常会导致算法收敛非常缓慢。在本文中首先讨论了均匀采样策略导致收敛缓慢的原因,并研究了通过在已有的推荐算法中融合内容信息改进采样策略并最终提高推荐效果的方法。实验证明,相比于均匀采样策略,通过融合内容信息的适应性采样策略的确能够加快推荐算法的学习。

\vskip 10bp

\hspace{5bp} {\zihao{-4}\textbf{【 关键词】}} 
推荐系统; 协同过滤; 适应性采样  
}